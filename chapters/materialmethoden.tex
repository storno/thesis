\chapter{Material \& Methoden}\label{chapter:materialmethoden}

\section{Plasmide}
Die folgenden Plasmide stammen entweder aus dem Laborbestand oder wurden von NEB GmbH (Frankfurt a. M.) erworben. Sie wurden unverändert transfiziert.

\begin{table}[htsb]
  \begin{tabular}{l l l l}
    \toprule
      A & B & C & D \\
    \midrule
      1 & 2 & 1 & 2 \\
      2 & 3 & 2 & 3 \\
    \bottomrule
  \end{tabular}
\end{table}

In den angegebenen Vektor wurden folgende Inserts kloniert. Dazu wurde die Methode der homologen Rekombination als Teil der Gateway-Technologie (Invitrogen, Karlsruhe) verwendet.

\begin{table}[htsb]
    \begin{tabular}{lll}
        \toprule
        Vector		&	Insert		& 	Reference	\\
        \midrule
        pSNAPf		&				&	New England Biolabs GmbH (Frankfurt a. M.)\\
        pCLIPf		&				&	New England Biolabs GmbH (Frankfurt a. M.)\\
        pSNAPf		&	ADRB2-16Gly	&	New England Biolabs GmbH (Frankfurt a. M.)\\
        pDONR221	&	ADRB2-16Arg	    &	IPT (TU München)\\
    \bottomrule
    \end{tabular}
\end{table}

In den angegebenen Vektor wurden folgende Inserts kloniert. Dazu wurde die Methode der homologen Rekombination als Teil der Gateway-Technologie (Invitrogen, Karlsruhe) verwendet.

\begin{table}[htsb]
\begin{tabular}{lll}
\toprule
Vektor		&	Insert		&	Polymorphismus / Mutation\\
\midrule
pSNAPf		&	5mis-ADRB2	&	Arg16, Tyr284\\
pSNAPf		&	ADRB2		&	Arg16\\
pCLIPf		&	ADRB2		&	Arg16\\
\bottomrule
\end{tabular}
\end{table}

\section{Bakterien}

\begin{tabular}{ll}
\toprule
Name		    &	Referenz\\
\midrule
E. coli (DH10B)	&    IPT (TU München)\\
\bottomrule
\end {tabular}


\section{Zelllinien}
text\\

\begin{tabular}{lll}
\toprule
Name		&	Source (Organ)				&	Reference\\
\midrule
HEK293		&	Human Embryonic Kidney		&	IPT (TU München)\\
HeLa		&	Human Cervix Epithelium		&	IPT (TU München)\\
\bottomrule
\end {tabular}\\

Based on the specified cell lines, the following stably expressing cell lines were generated:\\

\begin{tabular}{lll}
\toprule
Name		&	Stably Overexpressed Protein	&	Polymorphic Variants\\
\midrule
HEK293		&	$\beta_2$-adrenoreceptor		&	Arg16, Gly16, Tyr284\\
HeLa		&	$\beta_2$-adrenoreceptor		&	Arg16, Gly16, Tyr284\\
\bottomrule
\end{tabular}

\section{Chemicals \& Reagents}
If not specified otherwise, all chemicals and reagents were obtained from Applichem (Darmstadt), Carl Roth (Karlsruhe), Merck (Darmstadt) and Sigma-Aldrich (Taufkirchen). \\

\begin{tabular}{lll}
\toprule
Name							&	Company\\
\midrule
BG-Alexa488						&	New England Biolabs GmbH (Frankfurt a. M.)\\
BG-d2							&	Cisbio Bioassys (Codolet, France)\\
BG-Terbium						&	Cisbio Bioassys (Codolet, France)\\
\bottomrule
\end {tabular}

\section{Enzyme}

\begin{tabular}{lll}
\toprule
Name							&	Company\\
\midrule
DNA Ligase T4						&	New England Biolabs (Frankfurt a. M.)\\
DNA Polymerase AccuPrime \textit{Pfx}	&	Invitrogen (Karlsruhe)\\
DNA Polymerase Quikchange Lightning	&	Agilent Technologies (Waldbronn)\\
Restriction Endonucleases			&	New England Biolabs (Frankfurt a. M.)\\
Restriction Enzyme DpnI				&	Agilent Technologies (Waldbronn)\\
\bottomrule
		
\end {tabular}


\section{Oligonukleotidprimer}

\begin{tabular}{lll}
\toprule
Name		&	Sequenz 					&	Produkt\\
\midrule
ADRB2-SbfI-for	&&\\
ADRB2-XhoI-rev	&	AAA AAA CCT GCA GGC GGG CAA CCC GGG AAC GG	&	\textit{SbfI}-\textbf{ADRB2}-\textit{XhoI}\\
ADRB2-c850t\_t851a\_for	& 	CAT GGG CAC TTT CAC CTA CTG CTG GCT GCC CTT C & ADRB2(Tyr284)\\
ADRB2-c850t\_t851a\_rev	&	GAA GGG CAG CCA GCA GTA GGT GAA AGT GCC CAT G &\\
\bottomrule
\end{tabular}

\section{Pharmaka}
\begin{tabular}{lll}
\toprule
Name			&	Type										&	Company\\
\midrule
Alprenolol		&	$\beta_2$\-Adrenorezeptoragonist				&	Sigma-Aldrich GmbH\\
ICI-118,551		&	$\beta_2$\-inverser Adrenorezeptoragonist		&	Sigma-Aldrich GmbH\\
Isoproterenol	&	$\beta_2$\-Adrenorezeptoragonist				&	Sigma-Aldrich GmbH\\
Epinephrine		&	natürlihcer Adrenorezeptoragonist			& 	Sigma-Aldrich GmbH\\
\bottomrule

\end{tabular}