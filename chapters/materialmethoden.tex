\chapter{Material \& Methoden}\label{chapter:materialmethoden}

\section{Plasmide}
Die folgenden Plasmide stammen entweder aus dem Laborbestand oder wurden von NEB GmbH (Frankfurt a. M.) erworben. Sie wurden unverändert transfiziert.

\begin{table}[htsb]
    \begin{tabular}{lll}
        \toprule
        Vektor		&	Insert		& 	Referenz	\\
        \midrule
        pSNAPf		&				&	New England Biolabs GmbH (Frankfurt a. M.)\\
        pCLIPf		&				&	New England Biolabs GmbH (Frankfurt a. M.)\\
        pSNAPf		&	ADRB2-16Gly	&	New England Biolabs GmbH (Frankfurt a. M.)\\
        pDONR221	&	ADRB2-16Arg	    &	IPT (TU München)\\
    \bottomrule
    \end{tabular}
\end{table}

In die in der folgenden Tabelle angegebenen Vektoren wurden die beschriebenen Inserts kloniert. Dazu wurde die Methode der homologen Rekombination als Teil der Gateway-Technologie (Invitrogen, Karlsruhe) verwendet.

\begin{table}[htsb]
\begin{tabular}{lll}
\toprule
Vektor		&	Insert		&	Polymorphismus / Mutation\\
\midrule
pSNAPf		&	5mis-ADRB2	&	Arg16, Tyr284\\
pSNAPf		&	ADRB2		&	Arg16\\
pCLIPf		&	ADRB2		&	Arg16\\
\bottomrule
\end{tabular}
\end{table}

\section{Bakterien}
Zur DNA-Amplifikation wurde folgender Bakterienstamm verwendet.
\begin{table}[htsb]
\begin{tabular}{ll}
\toprule
Name		    &	Referenz\\
\midrule
E. coli (DH10B)	&    IPT (TU München)\\
\bottomrule
\end {tabular}
\end{table}


\section{Zelllinien \& Zellkultur}
Folgende Zelllinien wurden zur Transfektion verwendet.
\begin{table}[htsp]
	\begin{tabular}{lll}
\toprule
Name		&	Ursprung (Organ)				&	Referenz\\
\midrule
HEK293		&	menschliches, embryonales Nierenepithel		&	IPT (TU München)\\
HeLa		&	menschliches Cervixepithel		&	IPT (TU München)\\
\bottomrule
\end {tabular}
\end{table}

Basierend auf den angegebenen Zelllinien wurden folgende stabile Zelllinien mit der Methode der stabilen Transfektion generiert.

\begin{table}[htsb]
\begin{tabular}{lll}
\toprule
Name		&	Stabil überexprimiertes Protein	&	Polymorphismen des Proteins\\
\midrule
HEK293		&	$\beta_2$-adrenoreceptor		&	Arg16, Gly16, Tyr284\\
HeLa		&	$\beta_2$-adrenoreceptor		&	Arg16, Gly16, Tyr284\\
\bottomrule
\end{tabular}
\end{table}

\section{Chemikalien \& Reagenzien}
Falls nicht anders angegeben, wurden alle Chemikalien und Reagenzien von Applichem (Darmstadt), Carl Roth (Karlsruhe), Merck (Darmstadt) und Sigma-Aldrich (Taufkirchen) bezogen. Folgende SNAP-Substrate wurden wie angegeben bezogen.

\begin{table}[htsb]
\begin{tabular}{lll}
\toprule
Name							&	Company\\
\midrule
BG-Alexa488						&	New England Biolabs GmbH (Frankfurt a. M.)\\
BG-d2							&	Cisbio Bioassays (Codolet, France)\\
BG-Lumi4						&	Cisbio Bioassays (Codolet, France)\\
\bottomrule
\end {tabular}
\end{table}

\section{Enzyme}

\begin{tabular}{lll}
\toprule
Name							&	Company\\
\midrule
DNA Ligase T4						&	New England Biolabs (Frankfurt a. M.)\\
DNA Polymerase AccuPrime \textit{Pfx}	&	Invitrogen (Karlsruhe)\\
DNA Polymerase Quikchange Lightning	&	Agilent Technologies (Waldbronn)\\
Restriction Endonucleases			&	New England Biolabs (Frankfurt a. M.)\\
Restriction Enzyme DpnI				&	Agilent Technologies (Waldbronn)\\
\bottomrule
		
\end {tabular}


\section{Oligonukleotidprimer}

\begin{tabular}{lll}
\toprule
Name		&	Sequenz 					&	Produkt\\
\midrule
ADRB2-SbfI-for	&&\\
ADRB2-XhoI-rev	&	AAA AAA CCT GCA GGC GGG CAA CCC GGG AAC GG	&	\textit{SbfI}-\textbf{ADRB2}-\textit{XhoI}\\
ADRB2-c850t\_t851a\_for	& 	CAT GGG CAC TTT CAC CTA CTG CTG GCT GCC CTT C & ADRB2(Tyr284)\\
ADRB2-c850t\_t851a\_rev	&	GAA GGG CAG CCA GCA GTA GGT GAA AGT GCC CAT G &\\
\bottomrule
\end{tabular}

\section{Pharmaka}
\begin{table}[htsb]
\begin{tabular}{lll}
\toprule
Name			&	Type										&	Company\\
\midrule
Alprenolol		&	$\beta_2$\--Adrenorezeptoranagonist				&	Sigma-Aldrich GmbH\\
ICI-118,551		&	inverser $\beta_2$\--Adrenorezeptoragonist		&	Sigma-Aldrich GmbH\\
Isoproterenol	&	$\beta_2$\--Adrenorezeptoragonist				&	Sigma-Aldrich GmbH\\
Epinephrin		&	natürlicher Adrenorezeptoragonist			& 	Sigma-Aldrich GmbH\\
\bottomrule
\end{tabular}
\end{table}

\section{Molekularbiologische Methoden}
\subsection{DNA-Amplifikation mittels Polymerasekettenreaktion (PCR)}
Zur Amplifikation von kodierender DNA wurde die Methode der Polymerasekettenreaktion (PCR) mittels des Enzyms AccuPrime \textit{pfx} DNA Polymerase verwendet. Dabei wurde folgender Reaktionsmix vorbereitet.

\begin{table}[htsb]
\begin{tabular}{ll}
cDNA oder Plasmid--DNA 					& 100\si{\nano\gram}\\
Vorwärtsprimer							& 20\si{\pico\mol}\\
Rückwärtsprimer							& 20\si{\pico\mol}\\
AccuPrime \textit{pfx} Reaktionspuffer 	& 5\si{\micro\liter}\\
AccuPrime \textit{pfx} DNA Polymerase	& 1\si{\micro\liter}\\
ddH$_2$O								& ad 50\si{\micro\liter}\\
\end{tabular}
\end{table}
Der Reaktionsmix wurde nach folgendem Protokoll in einem Mastercycler Pro (Eppendorf, Hamburg) zur DNA-Amplifikation inkubiert.

\begin{tabular}{llll}
\toprule
 					& Temperatur 		& Dauer				& Zyklen\\
\midrule
Initiale Denaturierung		& 95\si{\celsius}	& 2\si{\minute}		& 1\\
\midrule
Denaturierung				& 95\si{\celsius}	& 15\si{\second}		& \\
Annealing					& 57\si{\celsius}	& 30\si{\second}		& 35\\
Elongation					& 68\si{\celsius}	& 1\si{\minute/kb}	& \\
\midrule
Finale Elongation 			& 68\si{\celsius}	& 1\si{\minute}		& 1\\
\bottomrule
\end{tabular}

\subsection{Agarose--Gelelektrophorese}

\begin{table}[htsb]
\begin{tabular}{lll}
50xTAE--Puffer: 	& Tris					& 0,2\si{M}\\
					& Essigsäure (0,5M)		& 57,1\si{\milli\liter}\\
					& Na$_2$EDTA x 2H$_2$O	& 37,2\si{\milli\liter}\\
					& ddH$_2$O				& ad 1\si{\liter}\\
					&				& \\
5xDNA--Ladepuffer: 	& Xylencyanol	& 0,025\si{\gram}\\
					& EDTA (0,5M)	& 1,4\si{\milli\liter}\\
					& Glycerol		& 3,6\si{\milli\liter}\\
					& ddH$_2$O		& 7,0\si{\milli\liter}\\
\end{tabular}
\end{table}
Die Herstellung eines einprozentigen Agarosegels erfolgte mit 1\si{\gram} Agarose in 100\si{\milli\liter} 1xTAE--Puffer, die durch Erhitzen in einer Mikrowelle gelöst wurde. Nach Abkühlen auf etwa 45\si{\celsius} wurden 6,5\si{\micro\liter} Ethidiumbromid hinzugefügt und das Gel mit den gewünschten Kämmen gegossen. Nach dem Auspolymerisieren wurde das Gel in eine mit 1xTAE--Puffer befüllte Elektophoresekammer (Peqlab, Erlangen) transferiert. Die DNA-Proben wurden 1:5 mit DNA-Ladepuffer verdünnt und in die Geltaschen geladen.