\chapter{Material \& Methoden}\label{chapter:materialmethoden}

\section{Material}
\subsection{Plasmide} \label{plasmide}
Die folgenden Plasmide stammen entweder aus dem Laborbestand (IPT (TU München))oder wurden von New England Biolabs GmbH (Frankfurt a. M.) erworben. Sie wurden unverändert transfiziert bzw. für weitere Klonierung verwendet.

\begin{table}[htsb]
    \begin{tabular}{lll}
        \toprule
        Vektor		&	Insert						& 	Referenz	\\
        \midrule
        pSNAPf		&								&	New England Biolabs GmbH (Frankfurt a. M.)\\
        pCLIPf		&								&	New England Biolabs GmbH (Frankfurt a. M.)\\
        pSNAPf		&	5mis--\textit{ADRB2}-16Gly	&	New England Biolabs GmbH (Frankfurt a. M.)\\
        pCLIPf		&	5mis--\textit{ADRB2}-16Gly	&	New England Biolabs GmbH (Frankfurt a. M.)\\
        pDONR221	&	\textit{ADRB2}-16Arg	   		&	IPT (TU München)\\
    \bottomrule
    \end{tabular}
\end{table}

In die in der folgenden Tabelle aufgeführten Vektoren wurden die angegebenen Inserts kloniert. Dazu wurde die Methode der homologen Rekombination als Teil der Gateway-Technologie (Invitrogen GmbH, Karlsruhe) verwendet.

\begin{table}[htsb]
\begin{tabularx}{\textwidth}{lll}
\toprule
Vektor		&	Insert		&	Polymorphismus / Mutation\\
\midrule
pSNAPf		&	5mis--\textit{ADRB2}	&	Arg16, Tyr284\\
pSNAPf		&	\textit{ADRB2}		&	Arg16\\
pCLIPf		&	\textit{ADRB2}		&	Arg16\\
\bottomrule
\end{tabularx}
\end{table}

\subsection{Bakterien}
Zur DNA-Amplifikation wurde der Bakterienstamm E. coli (DH10B) (IPT, TU München) verwendet.

\subsection{Zelllinien \& Zellkultur}
Folgende Zelllinien wurden zur Transfektion verwendet.

\begin{table}[htsp]
	\begin{tabular}{lll}
\toprule
Name		&	Ursprung (Organ)				&	Referenz\\
\midrule
HEK293		&	menschliches, embryonales Nierenepithel		&	IPT (TU München)\\
HeLa		&	menschliches Cervixepithel					&	IPT (TU München)\\
\bottomrule
\end {tabular}
\end{table}

Basierend auf den angegebenen Zelllinien wurden folgende stabile Zelllinien generiert. Die mit \textit{5-mis-SNAP} gekennzeichneten exprimierten Rezeptoren trugen am N-terminalen Ende einen vom 5HT$_3$-Rezeptor abgeleitete Membraninsertionssequenz sowie direkt C-terminal anschließend den SNAP-Tag (SNAP-tag, New England Biolabs GmbH, Frankfurt a. M.). 

\begin{table}[htsb]
\begin{tabularx}{\textwidth}{lll}
\toprule
Name		&	Stabil überexprimiertes Protein	&	Polymorphismen des Proteins\\
\midrule
HEK293		&	5mis-SNAP-ADRB2		&	Arg16, Gly16, Tyr284\\
HeLa		&	5mis-SNAP-ADRB2		&	Arg16, Gly16, Tyr284\\
\bottomrule
\end{tabularx}
\end{table}

\subsection{Chemikalien \& Reagenzien}
Falls nicht anders angegeben, wurden alle Chemikalien und Reagenzien von Applichem (Darmstadt), Carl Roth (Karlsruhe), Merck (Darmstadt) und Sigma-Aldrich (Taufkirchen) bezogen. Folgende SNAP-Substrate wurden wie angegeben bezogen.

\begin{table}[htsb]
\begin{tabular}{lll}
\toprule
Name							&	Company\\
\midrule
BG-Alexa488						&	New England Biolabs (Frankfurt a. M.)\\
BG-d2							&	Cisbio Bioassays (Codolet, France)\\
BG-Lumi4						&	Cisbio Bioassays (Codolet, France)\\
\bottomrule
\end {tabular}
\end{table}

\subsection{Enzyme}

\begin{tabularx}{\textwidth}{lll}
\toprule
Name							&	Referenz\\
\midrule
DNA Ligase T4							&	New England Biolabs (Frankfurt a. M.)\\
DNA Polymerase AccuPrime \textit{Pfx}	&	Invitrogen (Karlsruhe)\\
DNA Polymerase Quikchange Lightning		&	Agilent Technologies (Waldbronn)\\
Restriktionsendonukleasen				&	New England Biolabs (Frankfurt a. M.)\\
\bottomrule
		
\end {tabularx}


\subsection{Oligonukleotidprimer}
Die Oligonukleotidprimer wurden entweder von Eurofins MWG Biotech (Ebersberg) oder Sigma-Aldrich (München) synthetisiert. Sie wurden in bidestilliertem (dd) H$_2$O gelöst und auf 1\si{\milli M} eingestellt. 
 
\begin{tabularx}{\textwidth}{lXl}
\toprule
Name		&	Sequenz 					&	Produkt\\
\midrule
ADRB2-SbfI-for	&&\\
ADRB2-XhoI-rev	&	AAA AAA CCT GCA GGC GGG CAA CCC GGG AAC GG	&	\textit{SbfI}-\textbf{ADRB2}-\textit{XhoI}\\
ADRB2-c850t\_t851a\_for	& 	CAT GGG CAC TTT CAC CTA CTG CTG GCT GCC CTT C & ADRB2(Tyr284)\\
ADRB2-c850t\_t851a\_rev	&	GAA GGG CAG CCA GCA GTA GGT GAA AGT GCC CAT G &\\
\bottomrule
\end{tabularx}

\subsection{Pharmaka}
\begin{tabularx}{\textwidth}{lll}
\toprule
Name			&	Typ										&	Referenz\\
\midrule
Alprenolol		&	$\beta_2$\--Adrenorezeptoranagonist				&	Sigma-Aldrich GmbH\\
ICI-118,551		&	inverser $\beta_2$\--Adrenorezeptoragonist		&	Sigma-Aldrich GmbH\\
Isoproterenol	&	$\beta_2$\--Adrenorezeptoragonist				&	Sigma-Aldrich GmbH\\
Epinephrin		&	natürlicher Adrenorezeptoragonist				& 	Sigma-Aldrich GmbH\\
\bottomrule
\end{tabularx}

\section{Molekularbiologische Methoden}
\subsection{DNA-Amplifikation mittels Polymerasekettenreaktion (PCR)}
Zur Amplifikation von kodierender DNA wurde die Methode der Polymerasekettenreaktion (PCR) mittels des Enzyms AccuPrime \textit{pfx} DNA Polymerase verwendet. Dabei wurde folgender Reaktionsmix vorbereitet.

\begin{table}[htsb]
\begin{tabular}{ll}
cDNA oder Plasmid--DNA 					& 100\si{\nano\gram}\\
Vorwärtsprimer							& 20\si{\pico\mol}\\
Rückwärtsprimer							& 20\si{\pico\mol}\\
AccuPrime \textit{pfx} Reaktionspuffer 	& 5\si{\micro\liter}\\
AccuPrime \textit{pfx} DNA Polymerase	& 1\si{\micro\liter}\\
ddH$_2$O								& ad 50\si{\micro\liter}\\
\end{tabular}
\end{table}
Der Reaktionsmix wurde nach folgendem Protokoll in einem Mastercycler Pro (Eppendorf, Hamburg) zur DNA-Amplifikation den angegebenen Zyklen ausgesetzt.

\begin{tabularx}{\textwidth}{lllc}
\toprule
 					& Temperatur 		& Dauer				& Zyklen\\
\midrule
Initiale Denaturierung		& 95\si{\celsius}	& 2\si{\minute}		& 1\\
\midrule
Denaturierung				& 95\si{\celsius}	& 15\si{\second}		& \\
Annealing					& 57\si{\celsius}	& 30\si{\second}		& 35\\
Elongation					& 68\si{\celsius}	& 1\si{\minute/kb}	& \\
\midrule
Finale Elongation 			& 68\si{\celsius}	& 1\si{\minute}		& 1\\
\bottomrule
\end{tabularx}

\subsection{Agarose--Gelelektrophorese}

\begin{tabularx}{\textwidth}{lll}
50xTAE--Puffer: 	& Tris					& 0,2\si{M}\\
					& Essigsäure (0,5M)		& 57,1\si{\milli\liter}\\
					& Na$_2$EDTA x 2H$_2$O	& 37,2\si{\milli\liter}\\
					& ddH$_2$O				& ad 1\si{\liter}\\
					&				& \\
5xDNA--Ladepuffer: 	& Xylencyanol	& 0,025\si{\gram}\\
					& EDTA (0,5M)	& 1,4\si{\milli\liter}\\
					& Glycerol		& 3,6\si{\milli\liter}\\
					& ddH$_2$O		& 7,0\si{\milli\liter}\\
\end{tabularx}
\\
Die Herstellung eines einprozentigen Agarosegels erfolgte mit 1\si{\gram} Agarose in 100\si{\milli\liter} 1xTAE--Puffer, die durch Erhitzen in einer Mikrowelle gelöst wurde. Nach Abkühlen auf etwa 45\si{\celsius} wurden 6,5\si{\micro\liter} Ethidiumbromid hinzugefügt und das Gel mit den gewünschten Kämmen gegossen. Nach dem Auspolymerisieren wurde das Gel in eine mit 1xTAE--Puffer befüllte Elektophoresekammer (Peqlab, Erlangen) transferiert. Die DNA-Proben wurden 1:5 mit DNA-Ladepuffer verdünnt und in die Geltaschen geladen. Parallel wurden vorverdaute DNA--Stücke bekannter Länge (1--10\si{kb}) in eine Geltasche geladen.

An die so beladene Gelkammer wurde eine Spannung von 120\si{\volt} für 45\si{\minute} angelegt. Die negativ geladenen DNA--Fragmente liefen abhängig von ihrer Länge mit unterschiedlicher Geschwindigkeit anodenwärts. Das sich im Gel befindliche Ethidiumbromid interkalierte in die doppelsträngige DNA.

Anschließend konnte unter UV-Licht (Wellenlänge 312\si{\nano\meter}) der Standard mit den gesuchten Fragmenten abgeglichen werden.

\subsection{Extraktion von DNA aus Agarosegelen}
Zur Extraktion von DNA--Fragmenten aus Agarosegelen wurde die gewünschte DNA--Bande mit einem sterilen Skalpell ausgeschnitten und mithilfe des QIAquick Gel Extraction Kit (Qiagen, Hilden) gemäß dem Protokoll des Herstellers extrahiert. Schließich wurde die DNA aus der Extraktionssäule mit 15\si{\micro\liter} ddH$_2$O eluiert.

\subsection{Restriktionsenzymverdau von PCR--Produkten und Plasmid--DNA}
PCR--Produkte und Plasmid--DNA wurden entsprechend den Empfehlungen des Enzymherstellers (New England Biolabs, Frankfurt a. M.) mit Restriktionsendonukleasen verdaut. Wenn möglich wurden die High-Fidelity--(HF)--Varianten der Restriktionsenzyme verwendet. 

Für den vollständigen Verdau von Plasmid-DNA wurden 1--2\si{U} des gewünschten Enzyms mit 1\si{\micro\gram} DNA für 2,5\si{\hour} bei 37\si{\celsius} inkubiert. Zur Überprüfung der DNA--Fragmentlänge nach Mini-Präparation durchlief die verdaute DNA erneut eine Gelelektrophorese. 

\subsection{Ligation von DNA--Fragmenten}

\subsection{Messung der DNA--Konzentrationen}
Die Bestimmung der Konzentration in Wasser gelöster DNA erfolgte mittels des Spektrophotometers ND--1000 (NanoDrop, Wilmington, USA) und der vom Hersteller gelieferten Software. Mithilfe der NanoDrop Software konnte über die Absorption bei 260\si{\nano\meter} die Konzentration der gelösten DNA und über den Quotienten der Absorption bei 260\si{\nano\meter} und 280\si{\nano\meter} die Reinheit der Probe bestimmt werden. Lag der Quotient über 1,8, konnte von einem hohen Reinheitsgrad, d.h. geringer Kontamination mit Proteinen, Phenol oder anderen Kontaminanten ausgegangen werden.

\subsection{DNA-Sequenzierung}
Die Sequenzierung von Plasmid--DNA und PCR--Produkten wurde durch Eurofins MWG Biotech (Ebersberg) durchgeführt. Dazu wurden 20\si{\micro\liter} in ddH$_2$O gelöster DNA der Konzentration 0,1\si{\micro\gram/\micro\liter} zur Sequenzierung gegeben und die Sequenz anschließend mit dem Tool MacVector (MacVector, Inc.) mit der erwarteten Basenfolge abgeglichen.

\subsection{Transformation elektrokompetenter DH10B--Bakterien}

\begin{tabularx}{\textwidth}{lll}
LB-Agar: 			& 1\% Bacto-Trypton		& 10\si{\gram}\\
					& 0,5\% Hefeextrakt		& 5\si{\gram}\\
					& 0,5\% NaCl				& 5\si{\gram}\\
					& Agar					& 15\si{\gram}\\
					& NaOH 1\si{M}			& 1\si{\milli\liter}\\
					
					& ddH$_2$O				& ad 1\si{\liter}\\
					&						&\\
LB-Medium: 			& 1\% Bacto-Trypton		& 10\si{\gram}\\
					& 0,5\% Hefeextrakt		& 5\si{\gram}\\
					& 0,5\% NaCl				& 5\si{\gram}\\
					& NaOH 1\si{M}			& 1\si{\milli\liter}\\
					& ddH$_2$O				& ad 1\si{\liter}\\
					&&\\
nach Resistenzgen des Plasmids:	& Ampicillin 100\si{\micro\gram/\milli\liter}&\\
								& Kanamycin 33\si{\micro\gram/\milli\liter}&\\

\end{tabularx}
\\

Zur DNA-Amplifikation von Plasmiden oder Ligationsprodukten wurden 0,5\si{\micro\liter} Plasmid--DNA bzw. 3\si{\micro\liter} Ligationsreaktion zu 50\si{\micro\liter} elektrokompetenten DH10B--Bakterien gegeben, in eine Küvette (Gene Pulser 0,1\si{\centi\meter} Cuvette, Bio-Rad GmbH, München) überführt und mit einem Elektroporationsgerät (MicroPulser, Bio-Rad GmbH, München) eine gepulste Spannung von initial 1,8\si{\kilo\volt} angelegt. Das Elektroporationsprodukt wurde sofort in ein 1,5\si{\milli\liter}--Reaktionsgefäß überführt und in einem Schüttelinkubator (Thermomixer, Eppendorf AG, Hamburg) für 1\si{\hour} bei 37\si{\celsius} und 350\si{rpm} inkubiert. Bei Plasmid-Amplifikation wurden verschiedene Verdünnungen um 1:10, bei Ligation das vollständige Bakterienvolumen auf Agarplatten ausgestrichen, die über das dem Resistenzgen des Vektors entsprechende Antibiotikum verfügten. Die Agarplatten wurden über Nacht bei 37\si{\celsius} inkubiert.

\subsection{Mini/Maxi--Kultur und Mini/Maxi--DNA--Aufreinigung}
Zur weiteren DNA-Ampflifikation im Rahmen einer Mini--Kultur wurde mittels einer sterilen Pipettenspitze eine einzelne Bakterienkolonie von einer Agarplatte aufgenommen und in ein Reaktionsgefäß mit 4\si{\milli\liter} LB-Medium und dem Vektor entsprechenden Antibiotikum (Ampicillin 100\si{\micro\gram/\milli\liter} bzw. Kanamycin 33\si{\micro\gram/\milli\liter}) abgeworfen.
In einem Schüttelinkubator (Thermoschüttler, Adolf-Kühner AG, Birsfelden) wurde die Kultur mindestens 6\si{\hour} oder über Nacht bei 37\si{\celsius} und 170\si{rpm} inkubiert.
\\
\\
Die im folgenden beschriebene DNA--Aufreinigung erfolgte unter Verwendung der Puffer des Plasmid Maxi Kits (Qiagen, Hilden).

\subsubsection{Mini--DNA--Aufreinigung}
1,5\si{\milli\liter} der Mini--Kultur wurden in einem 1,5\si{\milli\liter} Reaktionsgefäß für 15\si{\second} bei 15000\si{rpm} abzentrifugiert, der Überstand verworfen. Zur Vergrößerung des Pellets wurde der Zentrifugationsschritt nach erneuter Zugabe von 1,5\si{\milli\liter} der Mini--Kultur wiederholt.

Das Pellet wurde in 250\si{\micro\liter} Resuspensionspuffer (P1) zur Degradation der bakteriellen RNA aufgenommen und 5\si{\minute} bei Raumtemperatur inkubiert. Nach Zugabe von 250\si{\micro\liter} Lysispuffer (P2) wurden die Proben gemischt und erneut 5\si{\minute} bei Raumtemperatur inkubiert. Die so in alkalischem Niveau lysierten Zellen wurden mit 300\si{\micro\liter} auf 4\si{\celsius} gekühltem Neutralisierungspuffer (P3) neutralisiert und 5\si{min} auf Eis inkubiert.

Nach Zentrifugation (10\si{\minute}, 15000\si{rpm}, 4\si{\celsius}) wurde die im Überstand befindliche DNA in ein neues Reaktionsgefäß überführt und mit 750\si{\micro\liter} reinem Ethanol für 5\si{\minute} bei Raumtemperatur präzipitiert. Mit einem weiteren Zentrifugationsschritt (5\si{\minute}, 15000\si{rpm}, 4\si{\celsius}) wurde die DNA pelletiert, mit 750\si{\micro\liter} 70\%igem Ethanol gewaschen, nochmals eine Minute zentrifugiert, luftgetrocknet und in 10\si{\micro\liter} ddH$_2$O gelöst.
Die Konzentration der DNA wurde photometrisch bestimmt.
\\
\\
Nach Restriktionsverdau zur Integritätsprüfung der über Mini--Präparation amplifizierten Plasmide erfolgte, wenn nötig weitere Amplifikation mittels Maxi--Kultur. Sollten die Klone weiter amplifiziert werden, wurden 2ml der Mini--Kultur in sterile Erlenmeyer--Reaktionsgefäße mit 100\si{\milli\liter} LB-Medium sowie dem entsprechenden Antibiotikum überführt und wiederum über Nacht im Schüttelinkubator inkubiert.

\subsubsection{Maxi--DNA--Aufreinigung}
Die Maxi--Kulturen wurden bei 6000g für 10\si{\minute} bei 4\si{\celsius} abzentrifugiert (Zentrifuge 5810R, Eppendorf, Hamburg), der Überstand verworfen.
Die DNA--Aufreinigung der Maxi--Kultur erfolgte entsprechend den Vorgaben des Herstellers des  Plasmid Maxi Kits von Qiagen (Hilden). Dazu wurde nach Degradation der in der Bakterienkultur befindlichen RNAasen eine alkalische Zelllyse durchgeführt, der pH-Wert anschließend angepasst und in salzfreier Lösung bei passendem pH die negativ geladene DNA über eine Anionenaustauschersäule gebunden, eluiert und mit Isopropanol präzipitiert. Das luftgetrocknete DNA--Pellet wurde in 150\si{\micro\liter} ddH$_2$O gelöst und die Konzentration wie beschrieben bestimmt. 

\section{Methoden der eukaryotischen Zellbiologie}
\subsection{Kultivierung eukaryotischer Zelllinien} \label{Kultur}
HEK293-- und HeLa--Zellen wurden bei 5\%CO$_2$ und 37\si{\celsius} in folgendem Zellkulturmedium inkubiert:

\begin{table}[htsb]
\begin{tabularx}{\textwidth}{ll}
	DMEM+++	&	Dulbecco's modified eagle's medium (DMEM) (PAN-Biotech, Aidenbach)\\
			&	+ 1\% L-Glutamin\\
			&	+ 1\% Penicillin (10.000\si{U/\milli\liter}) / Streptomycin (10.000 \si{\micro\gram/\milli\liter})\\
			&	+ 10\% fötales Rinderserum (FBS)\\
	\\
	\multicolumn{2}{c}{Transfizierte Zelllinien wurden zusätzlich mit folgendem Antibiotikum kultiviert:}\\
	&	+ 0,4\si{\gram/\liter} Geniticin (G-418)\\
\end{tabularx}
\end{table}

Die Zelllinien wurden in Zellkulturschalen (Nunc, Langensebold) kultiviert. Beide Zelllinien verdoppelten ihre Zellzahl nach etwa 24\si{\hour} und wuchsen als adhärente Monolayer. Alle drei bis vier Tage wurden die Zellen gesplittet. Nach Absaugen des Mediums und einmaligem Waschen mit DPBS (Dulbecco's Phosphate Buffered Saline) wurden die Zellen mit Trypsin-EDTA-Lösung (0,5\si{\gram/\liter} Trypsin, 0,2\si{\gram/\liter} EDTA, Pan-Biotech, Aidenbach) für 1\si{\minute} bei 5\%CO$_2$ und 37\si{\celsius} inkubiert, die Trypsin-EDTA-Lösung abgenommen und die nun abgelösten Zellen in Kulturmedium resuspendiert. Sie wurden 1:8 in neue Kulturschalen mit vorgelegtem Medium gesät.

\subsection{Auftauen und Einfrieren von Zellen}
\begin{tabularx}{\textwidth}{ll}
	Einfriermedium 	& DMEM+++ (s. \ref{Kultur})		\\
					& + 20\% fötales Rinderserum (FBS) \\
					& + 10\% Dimethylsulfoxid (DMSO) \\
\end{tabularx}

Zum Auftauen von zuvor in flüssigem Stickstoff gefrorenen Zellen, wurden sie in Kryogefäßen (Sarstedt AG, Nümbrecht) in einem Wasserbad bei 37\si{\celsius} aufgetaut. Währenddessen wurden die beschriebenen Zellkulturmedien auf 37\si{\celsius} vorgewärmt und in Zellkulturschalen vorgelegt. Die aufgetaute Zellsuspension wurde zugegeben. Nach etwa vier Stunden, wenn die Zellen adhärent waren, wurde das Zellkulturmedium abgesaugt und durch frisches Medium ersetzt, das kein DMSO mehr enthielt.
\\
\\
Sollten kultivierte Zellen eingefroren werden, wurden sie mittels Trypsin-EDTA-Lösung von den Zellkulturschalen abgelöst (s. \ref{Kultur}), in einem 15\si{\milli\liter}--Reaktionsgefäß bei 1200\si{rpm} für 5\si{\minute} bei Raumtemperatur zentrifugiert und in auf 4\si{\celsius} gekühltem Einfriermedium resuspendiert. Jeweils 1\si{\milli\liter} der Suspension wurde in vorgekühlte Kryogefäße gefüllt und bei -20\si{\celsius} gefroren. Nach 24\si{\hour} wurden sie weiter auf -80\si{\celsius} gekühlt. Nach weiteren 24\si{\hour} konnten die gefrorenen Zellen in flüssigen Stickstoff überführt werden.

\subsection{Transiente Transfektion von HEK293-- und HeLa--Zellen mit Effectene} \label{transfektion}
Zur transienten Transfektion von HEK293- und HeLa--Zellen wurde Effectene (Qiagen, Hilden) benutzt. 
\\
\\
Zum Zeitpunkt der Transfektion waren die Zellen in 6-Well-Platten bzw. 6\si{\centi\meter}--Schalen zu 60-80\% konfluent kultiviert. Unmittelbar vor der Transfektion wurden die Zellen einmal mit PBS gewaschen, daraufhin die unten beschriebene Menge DMEM vorgelegt.
\\
\\
Zur Herstellung des Transfektionsansatzes wurden zu angegebene Volumina der zu transfizierenden Plasmid--DNA, EC--Puffer und Enhancer vermischt und fünf Minuten bei Raumtemperatur inkubiert.
Mit dem Enhancer wurden die DNA--Moleküle in einem durch den EC--Puffer korrekt eingestellten Puffersystem so zuerst kondensiert. Die anschließende Zugabe des Effectene--Reagenzes führte zur Komplexierung der DNA mit einem kationischen Lipid. Nach ausreichendem Mischen und zehnminütiger Inkubation konnte die so komplexierte DNA tropfenweise auf die zu transfizierenden Zellen gegeben werden. Die DNA--Moleküle konnten so in den Zellkern eingeschleust werden. Nun in großer Zahl im eukaryoten Zellkern vorhandene transfizierte DNA wurde von den Zellen abgelesen und führte zu einem ausreichend hohen Expressionslevel.
\\
\\
Mit dem verwendeten Reagenz konnte eine Effizienz von etwa 70\% erreicht werden.

\begin{table}[htsb]
\begin{tabularx}{\textwidth}{lll}
	\toprule
											& 6-Well-Platte			& 6\si{\centi\meter}--Schale\\
	\midrule
	DNA	(1\si{\micro\gram/\micro\liter})		& 0,4\si{\micro\gram}	& 1\si{\micro\gram}	\\
	EC--Puffer 								& 100\si{\micro\liter}	& 150\si{\micro\liter}	\\
	Enhancer 								& 3,2\si{\micro\liter}	& 8\si{\micro\liter}	\\
	\midrule
	&\multicolumn{2}{l}{1\si{\second} vortexen und 5\si{\minute} bei Raumtemperatur inkubieren}\\
	\midrule
	Effectene 								& 5\si{\micro\liter}		& 10\si{\micro\liter}	\\
	\midrule
	&\multicolumn{2}{l}{10\si{\second} vortexen und 10\si{\minute} bei Raumtemperatur inkubieren}\\
	\midrule
	DMEM									& 600\si{\micro\liter}	& 1\si{\milli\liter}	\\
	DMEM vorgelegt							& 1,5\si{\milli\liter}	& 4\si{\milli\liter}	\\			
	\bottomrule
\end{tabularx}
\end{table}

\subsection{Generierung stabil exprimierender HEK293-- und HeLa--Zelllinien}
Alle verwendeten Plasmide besaßen das Neomycin-Geniticin-Resistenzgen (Neo\textsuperscript{r}). Zellen, die die transfizierte Plasmid-DNA stabil in ihr Genom integriert hatten, konnten so mit Geniticin (G--418, Invitrogen, Karlsruhe) selektioniert werden.\\
\\
Mit den Plasmiden aus \ref{plasmide} wurden unter Verwendung der in \ref{transfektion} beschriebenen Methode HEK293- und HeLa--Zellen mit Effectene transfiziert. Danach wurden die überexprimierenden Zellen in 10\si{\centi\meter}--Zellkulturschalen ausgesät. 24\si{\hour} nach Transfektion wurde die Zellen täglich für ein bis zwei Wochen mit frischem Medium versorgt, das mit 0,8\si{\gram/\liter}Geniticin versetzt war. 

Als unter dem invertierten Mikroskop einzelne Kolonien erkennbar waren, wurden diese gepickt und in eigenen Zellkulturschalen ausgesäht. Die so entstandenen heterogen exprimierenden Zellen wurden für fluoreszenzoptische Untersuchungen herangezogen. 

\section{Mikroskopische Methoden}

\section{Fluoreszenzoptische Methoden}



