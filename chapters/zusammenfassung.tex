\chapter{Zusammenfassung}\label{chapter:zusammenfassung}

Mit dieser Arbeit sollte mittels der trFRET-Methode die Oligomerisierung des \gls{beta2} untersucht werden. Dazu waren mehrere Methoden zu etablieren und validieren. Zuerst konnten mittels SNAP-tag geeignete Expressionssysteme generiert werden, die die kovalente Bindung trFRET-kompatibler Fluorophore erlaubten (s. Abschnitt \ref{klonierung}). Darauf gelang der Nachweis von Rezeptoroligomeren und damit zusammenhängenden Modulationen. Hierauf soll in Abschnitt \ref{discussion:limits} weiter eingegangen werden. Nach dem prinzipiellen Nachweis konnte der nicht-modifizierte \gls{beta2} weiter charakterisiert werden: trFRET zwischen bislang nicht verfügbaren Rezeptorliganden höchster Spezifität zeigte Rezeptoroligomere auf transfizierten HEK-Zellen. Mit der Etablierung der Methoden ergeben sich für zukünftige Anwendungen essentielle Grundlagen sowie die Möglichkeit des Transfers auf andere Rezeptoren.
