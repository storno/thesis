\chapter{Zusammenfassung}\label{chapter:zusammenfassung}

Entgegen dem klassischen Paradigma, die Gruppe der G-Protein-gekoppelten Rezeptoren sei strukturell auf Einzeleinheiten (Monomeren) basiert, geht man heute vom Vorhandensein konstitutiver Rezeptoroligomeren als erstem Signaltransduktor bei der zellulären Signalisierung aus. Zwar existierten bislang  eine Reihe von Untersuchungsmethoden, die diese These stützten, doch fehlten zur substantiellen Validierung noch Systeme, die unter in-vivo-Bedingungen spezifische Evidenz liefern können.

Mit dieser Arbeit konnte mittels einer auf tr-FRET basierenden Methode speziell die Oligomerisierung des \gls{beta2} untersucht werden. Dazu wurden zunächst mehrere Untersuchungsmethoden etabliert und validiert. 

Zuerst konnten unter Verwendung des SNAP-tag geeignete Expressionssysteme generiert werden, die die kovalente Bindung tr-FRET-kompatibler Fluorophore erlaubten. Mit dieser Methode gelang der Nachweis von Rezeptoroligomeren und die Untersuchung von modulierenden Faktoren. Es konnte gezeigt werden, dass der zellmembrangebundene ADRB2 Oligomere mindestens der Größenordnung zwei bildet.

 Weiter zeigte die Bindung von den ADRB2 adressierenden Agonisten, nicht jedoch die von Antagonisten oder inversen Agonisten einen signifikanten Signalanstieg des tr-FRET-Signals zwischen oligomerisierten Rezeptoren. Es ist davon auszugehen, dass die Anhebung des Signals nicht mit einer denkbaren Änderung der Anzahl der sich in einem Rezeptoroligomer befindlichen Monomere in Verbindung gebracht werden kann, sondern mit N-terminalen Konformationsänderungen der untersuchten Rezeptorspezies im Rahmen von Internalisierung unter Agonistenstimulation einhergeht. 
 
 Durch die ausbleibende Signaländerung bei Stimulation mit einem inversen Agonisten konnte dieser als in Bezug auf die Rezeptoroligomerisierung neutral eingestuft und für die folgende Ligandensynthese priorisiert werden. 
 
 Weiter konnte gezeigt werden, dass weder den bekannten Polymorphismen noch der Rezeptorglykosylierung des ADRB2 im Zusammenhang mit der Oligomerisierung eine Rolle zukommt.

 Nach dem prinzipiellen Nachweis der Oligomerisierung konnte der nicht-modifizierte \gls{beta2} mittels fluoreszierender Liganden weiter charakterisiert werden. Dazu wurden geeignete Liganden zur Synthese in Auftrag gegeben. Mit diesen bislang nicht verfügbaren invers antagonistischen Rezeptorliganden höchster Spezifität konnte auf membrangebundene Rezeptoroligomere auf transfizierten HEK-Zellen analog zu den zuvor untersuchten geschlossen werden. 
 
 Damit konnte die prinzipielle Eignung eines tr-FRET basierten Ansatzes im Falle des ADRB2 auch für native Zellen gezeigt werden. Dem Institut steht somit fortan ein leistungsfähiges Instrumentarium für weitere Untersuchungen zur Verfügung.
 
 Mit der Etablierung der Methoden ergeben sich für zukünftige Anwendungen essentielle Grundlagen sowie die Möglichkeit des Transfers auf andere Rezeptoren.
