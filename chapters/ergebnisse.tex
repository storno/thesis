\chapter{Ergebnisse} \label{chapter:ergebnisse}
\section{Generierung von $\beta_2$-Adrenorezeptoren mit dem SNAP-tag}
Zur Untersuchung der Oligomerisierung des \gls{beta2} wurden die Rezeptoren so modifiziert, dass sie über eine extrazelluläre Komponente verfügten, die Untersuchungen mit fluoreszierenden Substraten ermöglichten. Der \gls{snap} ermöglicht über seine O\textsuperscript{6}-Alkylguanin-DNA-Alkyltransferase-Aktivität die kovalente Bindung nahezu beliebiger Moleküle. Die gewünschten Fluorophore müssen dazu eine O\textsuperscript{6}-Benzylguanin oder O\textsuperscript{6}-Alkylguanin-Gruppe tragen.

Darauf basierend wurden Vektoren kloniert, die den \gls{beta2} trugen, der N-terminal über den \gls{snap} verfügte. Initiale Fluoreszenzmikroskopie zeigte, dass mit der N-terminalen Modifikation des \gls{beta2} keine Membranexpression des \gls{beta2} mehr erfolgte.

%TODO: Bilder: fehlende Expression ohne MIS 

%TODO: MIS 5-HT3A, Erzeugung der Vektoren

Infolgedessen wurde weiter N-terminal die Proteinsequenz zur Membraninsertion angefügt, die zur zufriedenstellenden Expression des \gls{beta2} führte. In Abbildung %TODO: Abbildungen Vektordesign mit den Polymorphismen
sind die final klonierten Vektoren dargestellt  
\section{Fluoreszenzmikroskopie des \gls{beta2} mit dem SNAP-tag}

\section{Oligomerisierung des \gls{beta2}}
\subsection{\gls{trfret} mit dem SNAP-Substraten}

\section{Einfluss der Ligandenstimulation auf die Oligomerisierung des \gls{beta2}}

\section{Einfluss der Glycosylierung auf die Oligomerisierung des \gls{beta2}}

\section{\gls{trfret} mit fluoreszierenden Liganden des \gls{beta2}}

